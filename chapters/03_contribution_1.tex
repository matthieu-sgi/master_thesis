\section{Plant becoming a sensor}

%TODO general : add bold

\subsection{Introduction}

The "State of the Art" section outlines the potential of plants to serve as bio-sensors by taking advantage of their natural responsiveness to environmental stimuli. It also explores the growing field of Human-Plant interaction, highlighting how plants' biological processes can be interpreted and integrated into technology-driven environments. This exploration provides a foundation for understanding how plants can bridge the gap between nature and technology.

In the following "Plant as a Sensor" section, this concept is taken further by focusing on the technical implementation of using plants as active sensors. This section introduces how specific sensor technologies are integrated with plants to capture environmental changes or human interaction, transforming these inputs into measurable data. Through this, plants are no longer passive elements but active participants in sensory networks, capable of real-time data collection and interaction, particularly in fields like Human-Computer Interaction and environmental monitoring.

\subsection{Technical choices}

Before diving into the implementation, several technical choices were made to shape the overall design of the system. The first major decision was selecting the ESP32 microcontroller as the core processing unit due to its versatility, offering built-in WiFi, multiple GPIO pins, and analog-to-digital converters, which allow for handling sensor data and wireless communication efficiently.

Additionally, the system's audio capabilities were a critical consideration, leading to the inclusion of a basic amplification circuit to boost sound output, ensuring compatibility with external speakers. The design also included a micro-SD slot to allow for expandable storage, enabling future enhancements like storing pre-recorded sounds or more detailed data logs. These foundational decisions laid the groundwork for a scalable and reliable system, ensuring smooth integration of the components that follow in the implementation.

In order to process efficiently the data on board, we aim to use the Estimated Mean Average (EMA) algorithm.

The Estimated Mean Average (EMA) is a statistical method that gives more weight to recent data points while still considering past observations. It is commonly used in time series analysis to track trends by smoothing out data fluctuations. Unlike the Simple Moving Average (SMA), the EMA adjusts more quickly to changes, making it useful in fields like finance, engineering, and sensor data analysis.

In finance, the EMA helps identify trends by reacting more sensitively to recent price changes. It is also applied in areas like environmental monitoring, where it processes sensor inputs to reflect real-time changes while minimizing short-term noise.

The formula for EMA is (ref \cite{hunter1986exponentially}):

\begin{math}
    EMA_t=\alpha⋅Y_t+(1-\alpha)⋅EMA_{t-1}

\end{math}

Where:
\begin{itemize}
    \item $EMA_t$ is the current EMA.
    \item $Y_t$ is the current data point.
    \item $\alpha$ is the smoothing factor.
    \item $EMA_{t-1}$ is the previous EMA.
\end{itemize}


The smoothing factor $\alpha$ controls the weight given to recent data, making EMA more responsive to recent changes while still considering past data.

In the IoT world, it can be useful to have those kind of models, because you can't keep the track of large amount of data. Indeed, the embedded have limited resources. The EMA allows to create a kind of mean with less data to store.


\subsection{The electronic interface}

The electronic interface is the interface that allows a compute unit to capture and interpret the plant signal and
communication. The interface is a device made by us for this use case. The printed circuit board (PCB)
device is composed of 3 main parts:
\begin{itemize}
    \item The core of the circuit, the microcontroller, an ESP32 Wroom 32
    \item An electronic filter connected using an electrode to the plant
    \item A sound part of the PCB that is including an audio amplifier, a volume knob and a terminal block to connect a speaker
\end{itemize}

The design of the PCB has been done using the open source software Kicad.
As said previously, the circuit contains 3 parts.

The core of the circuit is the computation part, including the microcontroller, an ESP32. All the other
devices of the circuit are connected to the ESP32. The choice to use a devkit has been done
to ease the electronic conception and to avoid any communication and soldering issue with the MCU\footnote[1]{Microcontroller Unit}.

\begin{figure}[h!]
    \centering
    \includegraphics[width=\textwidth]{images/iop.pdf}
    \caption{The main schematic of the device. This main schematic regroups components and sub-sheets (including the audio circuit
        and the filter circuit). The main component is an ESP32 Wroom DevKit. All other components are linked to this microcontroller}
    \vspace{0.1cm}
    \label{fig:iop_schematic_main}
\end{figure}


The PCB also includes a dedicated slot for a capacitive humidity sensor (ref figure \ref{fig:moisture_sensor}). This humidity sensor is added to reinforced the data captured by the main sensor/filter. This sensor captures the moisture level in the plant's soil, providing valuable insight into the plant's overall health and hydration status.

By integrating this additional sensor, the system can monitor environmental factors alongside the main sensor's data, such as the plant's response to touch or other interactions. The humidity data complements the primary sensor readings which can be used to refine the interaction model or influence the sonification process. This added layer of data could permits more precision and complexity when doing the sonification process.

\begin{figure}[h]
    \centering
    \includegraphics[width=0.8\textwidth]{moisture_sensor.jpg}
    \caption{The sensor chosen is the capacitive moisture sensor from Seeed Studio: The Grove. This sensor is able to capture the moisture level in the plant's soil, providing valuable insight into the plant's overall health and hydration status. However, the connector on the PCB allows to connect any other sensor that has 3 pins (VCC, GND, DATA) connection.}
    \vspace{0.1cm}
    \label{fig:moisture_sensor}
\end{figure}


The circuit component that allows us to read data from the plant is the electronic filter.
This filter has been designed by \textit{Jakub Nikonowicz} and \textit{Łukasz Matuszewski}
from \textit{Politechnika Poznańska}.
Thanks to them, I adapted it for my application on my embedded device.

\begin{figure}[H]
    \centering
    \includegraphics[width=\textwidth]{images/iop-plant_filter.pdf}
    \caption{The electronic circuit designed to capture the interaction by analyzing the electronic
        frequency response. The circuit includes 3 resistors, 3 inductors and 3 capacitors as main components}
    \vspace{0.1cm}
    \label{fig:iop_schematic_filter}
\end{figure}

This filter is ending by a crocodile clamp that is directly connected to the plant.

The last part of the circuit is the sound output/rendering. This circuit includes a small amplifier,
the LM386 from Texas Instruments. The rest of the circuit are components needed in order to
induce amplification on the signal without creating to many noise and saturation.

\begin{figure}[H]
    \centering
    \includegraphics[width=\textwidth]{images/iop-audio_circuit.pdf}
    \caption{The sound output part of the circuit that is used to render the sound.
        This part includes a small amplifier, the LM386. The circuit also includes the components necessary
        to control and handle the amplification (reduce noise and saturation)}
    \vspace{0.1cm}
    \label{fig:iop_schematic_audio}
\end{figure}

The PCB is built to be able to add a micro-SD slot (ref to \ref{fig:iop_schematic_main}). This micro-SD slot
allows to increase the storage of the embedded ROM. This can be used to store pre-built sounds and music to avoid generating a sound on-board.


Once the schematic was completed, the next step was routing the tracks. There are several methods for PCB routing, including single-sided, double-sided, and multiple-layer designs. We opted for a double-sided PCB on each side because this configuration simplifies component placement, making the layout more efficient and organized. Additionally, it is significantly more cost-effective compared to multi-layer PCBs, striking a good balance between performance and budget. Indeed, in terms of routing, PCBs can be single-sided, double-sided, or multi-layered. Single-sided boards are simple and low-cost but limited in complexity. Double-sided PCBs, with copper on both sides, allow for more efficient routing and are commonly used for moderate complexity. Multi-layered PCBs offer even greater routing density, suitable for advanced electronics such as computer motherboards or phone components.

In terms of track width, we used 0.2mm wide tracks for data signals to maintain signal integrity, while the power tracks were designed with a width of 0.8mm. This wider width ensures that the power tracks can safely handle up to 800 mA, providing sufficient current capacity without overheating or voltage drops.

\begin{figure}[H]
    \centering
    \includegraphics[width=\textwidth]{images/iop-routed_pcb.pdf}
    \caption{Double sided PCB routed. This view allows to see the track used for components inter-connection. The tracks width is 0.2mm for data tracks and 0.8mm for power tracks.}
    \vspace{0.1cm}
    \label{fig:iop_routed_pcb}
\end{figure}

Kicad also allows us to generated a 3D view of the future PCB. This allows us to imagine what the
PCB will look like when it will be manufactured.
\begin{figure}[H]
    \centering
    \includegraphics[width=0.8\textwidth]{images/front_iop_3D_view_modified.png}
    \caption{Front 3D rendering of the built PCB. The rendering is done using open source software: Kicad}
    \vspace{0.1cm}
    \label{fig:front_iop_3D_view_modified}
\end{figure}

\subsection{Human interaction}

\subsubsection{Use of the sensor/filter}

The device captures human interaction with the plant by detecting changes in its impedance, capacitance, and inductance. These changes are measured through GPIO 14 of the ESP32 DevKit, which reads analog data and converts them into digital values using a 12-bit analog-to-digital converter (ADC). The ADC outputs integer values ranging from 0 to 4095, and these values fluctuate depending on the interaction, such as touching, grasping, or tapping the plant.

In addition to capturing basic interaction data, the device employs a swept frequency technique that scans frequencies from 100 kHz to 300 kHz. This range allows for a more detailed measurement of the plant’s response to different types of interactions. The data from these swept frequencies is stored in an array, where each frequency corresponds to a specific index in the array.

To handle the fluctuating data and provide more consistent results, I implemented a normalization process using an Estimated Mean Average (EMA). Instead of normalizing the array based solely on its current values, the normalization is done by comparing the current values with the corresponding index from a previous version of the array. This use of EMA helps smooth out short-term noise while retaining the overall trend in the frequency response. The EMA provides a balance between responsiveness and stability, making it well-suited for capturing plant interactions in real-time.

To manage the calculations and data processing required for the swept frequencies and EMA, I developed a custom C library. This library handles the mathematical operations involved in calculating the EMA, normalizing the data, and managing the array of frequencies. To facilitate easier testing and development, I created Python bindings for the C library, enabling me to test and verify the functionality of the library in a more accessible and interactive environment. Using Python, I was able to rapidly prototype and test different configurations, ensuring that the C library operates efficiently before deploying it on the embedded system.

This combination of a custom C library for performance and Python bindings for rapid testing has allowed for both robust data processing on the ESP32 and flexibility during the development phase.

%TODO: Add a table comparing the values depending on the interaction

\subsubsection{Sonification on the device}

The human interaction goes through the sonification on the device. ESP32 embeds a 8-bit digital to analog converter (DAC).
The embed DAC is enough to play low quality sounds. Combined with the micro-SD card including in it, it is a media player.
Using the \textit{Arduino Audio Tools} library it is possible to read MP3, Wav and other audio format.
Taking the sensor value as an input, you can then apply a pitch shift on the data that will be rendered on the DAC.
The pitch shift is a value included between 0 and 100. We are mapping the max value of the sensor and the min value
to this range using the \textit{map()} Arduino function.

The data is sent to the DAC and rendered. \textit{Arduino Audio Tools} also includes a way to communicate data to the
DAC.

%TODO: Add oscilloscope screenshot of the sound output.

There are two DAC available on the ESP32. DAC 1 and DAC 2 respectively on GPIO25 and GPIO26. The IoP device allows
to choose between the one you want by using a removable jumper.

The output of the DAC is too low to be able to render it on a speaker. The amplification circuit allows a small
amplification of the output. The amplifier is a basic and simple one. I added a 10k Ohms potentiometer to be able to
tweak the volume. The sound quickly reach amplifier limitations and is becoming saturated.

The amplified output is sent to a terminal block that acts as the connection interface to the audio speaker.
The audio speaker chosen in our specific example is a 8 Ohms 3 Watts speaker.

\begin{figure}[h!]
    \centering
    \includegraphics[width=0.6\textwidth]{images/speaker.jpg}
    \caption{8 ohms 3 Watts basic speaker used in our example.}
    \vspace{0.1cm}
    \label{fig:speaker}
\end{figure}


\newpage
\subsubsection{User study}
\input{chapters/user_study}
\subsection{Evaluation of the system}

In evaluating the system for using plants as sensors, various interactions were analyzed to understand how they impact the signal output. The study demonstrated a clear difference in signal behavior when the plant was touched versus when it was not. For example, at a frequency of 100 kHz (ref figure \ref{fig:100_khz_signals}), the oscilloscope images showed a significant change in the signal, where the interaction caused a noticeable shift in the output waveform. This contrast was even more pronounced at higher frequencies, such as 300 kHz (ref figure \ref{fig:300_khz_signals}), which provided clearer signals, especially when using the trigger function on the oscilloscope. The system effectively captured the interaction, and the filtering process helped distinguish the touched state from the non-touched one. These results confirm the system's ability to respond to human interaction, providing accurate real-time data while using the exponential moving average (EMA) to smooth fluctuations.

\begin{figure}[h!]
    \centering
    \begin{minipage}{.5\textwidth}
        \centering
        \includegraphics[width=0.9\linewidth]{100_khz_no_touch.png}
        \label{fig:100_khz_no_touch}
    \end{minipage}%
    \begin{minipage}{.5\textwidth}
        \centering
        \includegraphics[width=.9\linewidth]{100_khz_touched.png}
        \label{fig:100_khz_touched}
    \end{minipage}
    \caption{Example of outputted signal from the filter with the plant touched and non-touched. The yellow signal is the signal generated from the ESP32. The violet one is the output of the filter, after interacting with the plant. The frequency is currently set at 100kHz to compared the 2 states. The left oscilloscope signal is the non-touched system. The right oscilloscope signal is the touched system. On these images, we did not use the trigger function from the oscilloscope.}
    \label{fig:100_khz_signals}
\end{figure}


\begin{figure}[h!]
    \centering
    \begin{minipage}{.5\textwidth}
        \centering
        \includegraphics[width=0.9\linewidth]{300_khz_no_touch.png}
        \label{fig:300_khz_no_touch}
    \end{minipage}%
    \begin{minipage}{.5\textwidth}
        \centering
        \includegraphics[width=.9\linewidth]{300_khz_touched.png}
        \label{fig:300_khz_touched}
    \end{minipage}
    \caption{Example of outputted signal from the filter with the plant touched and non-touched. The yellow signal is the signal generated from the ESP32. The violet one is the output of the filter, after interacting with the plant. The frequency is currently set at 300kHz to compared the 2 states. The left oscilloscope signal is the non-touched system. The right oscilloscope signal is the touched system. On these images, we used the trigger function from the oscilloscope, explaining the clean output.}
    \label{fig:300_khz_signals}
\end{figure}


\begin{table}[h!]
    \begin{tabular}{|l|lll|}
        \hline
        Frequency                & State     & Peak to peak (mV) & Mean voltage (mV) \\ \hline
        \multirow{2}{*}{100 kHz} & Untouched & 192               & 198.01            \\
                                 & Touched   & 168               & 152.02            \\ \hline
        \multirow{2}{*}{150 kHz} & Untouched & 260               & 534.93            \\
                                 & Touched   & 300               & 487.77            \\ \hline
        \multirow{2}{*}{200 kHz} & Untouched & 340               & 464.71            \\
                                 & Touched   & 260               & 259.56            \\ \hline
        \multirow{2}{*}{250 kHz} & Untouched & 340               & 506.18            \\
                                 & Touched   & 200               & 286.08            \\ \hline
        \multirow{2}{*}{300 kHz} & Untouched & 280               & 325.58            \\
                                 & Touched   & 160               & 188.03            \\ \hline
    \end{tabular}
    \caption{Data collected using the oscilloscope method on different frequencies. We are looking at the peak to peak voltage and the mean voltage on several periods.}
    \label{tab:sum_up_data_oscil}
\end{table}

In the evaluation of the "Plant as a Sensor" system, various frequencies were tested to assess the plant's response to touch and its effect on the signal output. A table (ref table \ref{tab:sum_up_data_oscil}) comparing peak-to-peak voltages and mean voltages across frequencies from 100 kHz to 300 kHz showed significant differences between the touched and untouched states of the plant.

For instance, at 100 kHz, the untouched plant produced a peak-to-peak voltage of 192 mV and a mean voltage of 198.01 mV, while the touched plant produced lower values, with 168 mV for peak-to-peak and 152.02 mV for mean voltage. Similar trends were observed at other frequencies, with the touched state consistently showing reduced voltage levels. At 250 kHz, for example, the untouched state had a peak-to-peak voltage of 340 mV, whereas the touched state was reduced to 200 mV, indicating a substantial difference caused by the interaction.

% \begin{figure}[h!]
%     \centering
%     \includegraphics[width=\textwidth]{esp32_signal.jpg}
%     \caption{Signals coming from the system. The yellow signal is the signal generated by the ESP32 (PWM). The purple one is the signal that we capture at the output of the filter. This signal is the one that goes in the Plant.}
%     \vspace{0.1cm}
%     \label{fig:esp_32_signal}
% \end{figure}

%TODO: Power consumption !


% 1. Technical Performance Testing

%     Accuracy of Plant Signal Capture: Measure how effectively the electronic filter captures changes in the plant's impedance, capacitance, and inductance when touched. This can be done by comparing sensor output data with known interactions to determine precision.
%     System Responsiveness: Test the responsiveness of the ESP32 microcontroller to human interaction with the plant. Evaluate the delay (latency) between the interaction and the corresponding sensor reading.
%     Signal Integrity: Assess the quality of the signals captured, ensuring that they are free from noise and distortion. Testing under varying environmental conditions (e.g., humidity, temperature) could be useful to ensure robustness.

% 2. User Study (Usability Testing)

%     Human Interaction Evaluation: Conduct controlled experiments to observe how users interact with different plant species, as detailed in the thesis. Participants can be asked to engage with the plants intuitively, and the interactions can be categorized as in the existing user study (e.g., pinch, grasp, slide). This helps evaluate the naturalness and intuitiveness of the plant as an interface.
%     User Satisfaction: Gather qualitative feedback from users on their experience interacting with the plants and the system's ability to translate their touch into meaningful sound. Surveys or interviews can help measure how engaging and satisfying the interaction feels.
%     Cross-species Comparison: Extend the existing user study to a wider variety of plants to ensure the system's adaptability across different plant types.

% 3. Sonification Quality

%     Audio Feedback Appropriateness: Evaluate the correlation between plant interaction and sound output. Test whether users perceive the generated sound as natural and whether different interactions (e.g., light touch vs. strong grasp) produce distinguishable auditory outputs. This could be assessed with a mixed-methods approach (e.g., user ratings and auditory analysis).
%     Sound Quality: Measure the quality of the sound generated by the embedded DAC and amplifier in terms of clarity, volume, and distortion.

% 4. System Stability and Scalability

%     Stress Testing: Test how the system behaves under continuous use and under varying loads (e.g., multiple touches in rapid succession). This will assess the system’s durability and resistance to breakdowns.
%     Scalability: Investigate whether the system can handle an increased number of sensor connections (additional plants) without performance degradation. This can be done by adding more plants and monitoring the system's response.

\subsection{Discussion}

The final product is an embedded device that include signal filtering, wireless communication and embedded sonification.


A better audio amplifier could be explored in order to reduce the distortion of the sound.
Adding an external digital to analog converter is also a possibility in order to upgrade the output. However, this possibility adds new components that will increase the size. Exploring the I2S protocol opens better output.
The I2S (Inter-IC Sound) protocol is used to transmit digital audio between devices like microcontrollers and DACs. It enables high-quality stereo audio transfer, using a master-slave setup to synchronize data with clock signals. Ideal for applications needing accurate sound transmission.

% Expand discussion


\subsection{Conclusion}

In conclusion, the standalone electronic system demonstrates significant potential in transforming natural plants into interactive bio-sensors, using the capabilities of a powerful microcontroller like the ESP32. The system's ability to capture plant responses to human touch and translate them into digital data is made possible by its efficient electronic interface. This architecture enables real-time interaction and offers new approach to human-plant interaction. This is also driving research into the use of plants' natural capacities as sensors. However, despite the microcontroller's strengths, the system still has notable limitations.

One major challenge lies in the sonification process, where the system struggles with producing high-quality, nuanced sound outputs due to the basic 8-bit DAC and limited audio amplification. Additionally, the data processing capabilities of the standalone device are limited, restricting the complexity of interactions it can detect and interpret. The sensor accuracy also has room for improvement, particularly in capturing fine-grained variations in plant interaction, which could taint the overall user experience and reduce the immersion.

To overcome these standalone limitations, a more sophisticated architecture is required. By connecting multiple devices in a distributed system, the Internet of Plants approach can enhance the processing power, improve sonification through external software, and allow for more complex data analysis. It could unlock the  full potential of plant-based sensors. The next section will explore how this expanded network can significantly enhance both the system's capabilities and user experience.
